\documentclass{article}
\usepackage{fancyhdr}
\usepackage{amsmath}
\usepackage{amsthm}

\pagestyle{fancy}
\setlength\headheight{28pt}
\addtolength{\headheight}{\baselineskip}
\fancyhf{}
\renewcommand{\headrulewidth}{0pt}
\lhead{Math 251 - Shuichi Masuda\\Juno Suárez\\\today}
\rfoot{Page \thepage}


\begin{document}


\section{Graded Problems \# 1}

\subsection{}

For the piecewise function

$$
f(x) =
\begin{cases}
  \pi & \textnormal{if } x \leq 0 \\
  3.1415926 & \textnormal{if } x > 0
\end{cases}
$$

\noindent
we're asked to investigate $ \lim_{x \to 0} f(x)$.
\\

We begin by asking if $ \lim_{x \to 0} f(x)$ exists. For the limit to exist at a point $c$, the right-hand limit and the left-hand limit must each exist at the $c$ and they must be equal, by the definition of limit.

The left-hand limit of $f(x)$ exists at $x=0$ because it is possible to get arbitrarily close to $0$ from the left. In fact, $f(x)$ is defined as a constant for all values on the left-hand side of, that is, less than $0$.
Therefore, the left-hand limit is equal to that constant value, namely, $\pi$.

Similarly, the right-hand limit of $f(x)$ exists at $x=0$, because $f(x)$ is defined as a constant for all values to the right of, or greater than, $x=0$. That constant is also the right-hand limit, $3.1415926$.

We have shown that left- and right-hand limits for $f(x)$ exist at $x=0$, however they are not equal, therefore $f(x)$ is not continuous at $x=0$.

While the decimal $3.1415926$ is a good enough approximation for $\pi$ in many
applications, it's not good enough for $f(x)$ to be continuous at $x=0$:
the values are not equal, therefore $\lim_{x \to 0} f(x)$ does not exist.




\subsection{}
See graph on cover sheet.

\end{document}
