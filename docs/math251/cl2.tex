\documentclass{article}
\usepackage{fancyhdr}
\usepackage{lastpage}
\usepackage{amsmath}
\usepackage{amsthm}
\usepackage[margin=1in]{geometry}
\pagestyle{fancy}
\setlength\headheight{28pt}
\addtolength{\headheight}{\baselineskip}
\fancyhf{}
\renewcommand{\headrulewidth}{0pt}
\lhead{Math 251 - Shuichi Masuda\\Juno Suárez\\\today}
\rfoot{Page \thepage \hspace{1pt} of \pageref{LastPage}}

\renewcommand\thesubsection{\arabic{subsection}.}
\renewcommand\thesubsubsection{\alph{subsubsection})}

\begin{document}

\section*{Computer Lab \# 2}

\subsection{}
For $$f(x) = \frac{sin(x)}{x}$$ $f(0)$ is not defined because $x$ appears in the
denominator. Since $x = 0$ would result in division by zero, $0$ is not in the
domain of $f(x)$, and $f(0)$ is not defined.

\subsection{}
\emph{Step performed in GeoGebra.}

\subsection{}
Graphically, $f(x)$ appears to approach $1$ at $x=0$. Intuitively, this seems right
because $sin(x)$ is periodic and $sin(0)$ is a local minimum for the period.
$sin(0)=0$, so values of $sin(x)$ near $sin(0)$ will be near 0, and
$\frac{near k}{k} \approx 1$ for non-zero values of $k$.

\subsection{} \label{numeric}
Numeric evaluation of $f(x)$ for $x$ near $0$, rounded to 5 decimal digits:
\\

\begin{tabular}{rl}
    $x$       & $f(x)$ \\
    0.1     & 0.95885  \\
    0.1     & 0.99833   \\
    0.01    & 0.99980   \\
    0.001   & 1.00000   \\
    -0.001  & 1.00000  \\
    -0.01   & 0.99980  \\
    -0.1    & 0.99833  \\
    -0.5    & 0.95885  \\
\end{tabular}

\subsection{}
In the table from \#\ref{numeric}, the values of $f(x)$ for $x = \pm 0.001$ round
to $1$ when rounded to 5 digits. This strongly supports our earlier conjecture that
$f(0)$ appears to be $1$.

\subsection{}
Yes, the observations in \#3 and \#5 are consistent.

\subsection{}
Since the behavior of $f(x)$ for values of $x$ near 0 appears to approach 0 both
graphically and numerically, it seems likely that $lim_{x \rightarrow 0} \frac{sin(x)}{x} = 0$.

\subsection{}
For the items in this section, I reused the graph and spreadsheet from steps \\
\#2 – \#5. I added a column to the spreadsheet for the limit value of $x$ such
that the values for $x$ would update to be values approaching the limit value of
$x$, and the third column would update with the calculated values of $f(x)$ for
each evaluated $x$.

Further, I defined new functions e.g. $a(x) = ...$ and defined
$f(x)=a(x)$, in an attempt to parameterize the function under investigation. I discovered
that the input bar also allows $f(x)=a$ and even $f=a$. Thus, for each function, I was able to investigate it graphically and numerically simply by redefining $f$ in the graphical view and updating the limit value for $x$ in the spreadsheet view.

If I were to continue improving the GeoGeobra environment for this investigation, I would like to graph a vertical line at the limit value of $x$ to better visualize the limit value of $f(x)$ at that point.

\subsubsection{}
\begin{flalign*} & \text{Conjecture: }
  \lim\limits_{x \rightarrow 9} \frac{x-9}{\sqrt{x}-3} = \boxed{6}
& \end{flalign*}

\subsubsection{}
\begin{flalign*} & \text{Conjecture: }
  \lim\limits_{x \rightarrow 4} x^2 = \boxed{16}
& \end{flalign*}

\subsubsection{}
\begin{flalign*} & \text{Conjecture: }
  \lim\limits_{x \rightarrow 0} sin\left(\frac{1}{x}\right) = \boxed{DNE}
& \end{flalign*}

\subsubsection{}
\begin{flalign*} & \text{Conjecture: }
  \lim\limits_{x \rightarrow 0} \frac{x}{|x|} = \boxed{DNE}
& \end{flalign*}

\subsubsection{}
\begin{flalign*} & \text{Conjecture: }
  \lim\limits_{x \rightarrow -3} 2 = \boxed{2}
& \end{flalign*}

\subsubsection{}
\begin{flalign*} & \text{Conjecture: }
  \lim\limits_{x \rightarrow 2} \frac{1}{x-2} = \boxed{DNE}
& \end{flalign*}


\end{document}
