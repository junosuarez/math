\documentclass{article}
\usepackage{fancyhdr}
\usepackage{lastpage}
\usepackage{amsmath}
\usepackage{amssymb}
\usepackage{amsthm}
\usepackage{tabstackengine}
\usepackage[margin=1in]{geometry}
\usepackage{graphicx}
\usepackage{textcomp}
\graphicspath{ {./} }
\TABstackMath
\TABbinary

\pagestyle{fancy}
\setlength\headheight{28pt}
\addtolength{\headheight}{\baselineskip}
\fancyhf{}
\renewcommand{\headrulewidth}{0pt}
\lhead{Math 251 - Shuichi Masuda\\Juno Suárez\\\today}
\rfoot{Page \thepage \hspace{1pt} of \pageref{LastPage}}

\renewcommand\thesubsection{\arabic{subsection}.}
\renewcommand\thesubsubsection{\alph{subsubsection})}

\begin{document}

\section*{Graded Problem \# 4}

\subsection{}
\subsubsection{}
See cover sheet. Graphically, the tangent lines for the ellipse and the parabola appear to be perpendicular at $P = (2,4)$.

\subsubsection{}
Let E be an ellipse given by the equation $2x^2 + y^2 = 24$ and let S be a sideways parabola given by the equation $y^2=8x$, and let point $P = (2,4)$. Then we can determine whether E and S are orthogonal at P analytically by implicitly differentiating E and S and computing the their derivatives at P. Since the values of the derivatives at P can be interpreted as the slope of the tangent line at P, these values can be compared to determine if they are perpendicular, and thus if E and S are orthogonal at P.

First we find $\frac{dy}{dx}$ for E using implicit differentiation:
\begin{align*}
  2x^2 + y^2 &= 24\\
  \frac{dy}{dx}(2x^2 + y^2) &= \frac{dy}{dx}(24)\\
  4x + 2y \cdot \frac{dy}{dx} &= 0\\
  \frac{dy}{dx} &= \boxed{ \frac{-2x}{y} }
\end{align*}

And evaluate at $P$:
\begin{align*}
  \left. \frac{dy}{dx} \right\rvert_{(2,4)} &= \frac{-2 \cdot 2}{4}\\
  \phantom{\left. \frac{dy}{dx} \right\rvert_{(2,4)}} &= \boxed{-1}\\
\end{align*}

Second, we repeat the procedure for S:
\begin{align*}
  y^2 &= 8x \\
  \frac{dy}{dx}(y^2) &= \frac{dy}{dx}(8x) \\
  2y \cdot \frac{dy}{dx} &= 8 \\
  \frac{dy}{dx} &= \boxed{\frac{4}{y}} \\
  \\
  \left. \frac{dy}{dx} \right\rvert_{(2,4)} &= \frac{4}{4} \\
  \phantom{\left. \frac{dy}{dx} \right\rvert_{(2,4)}} &= \boxed{1} \\
\end{align*}

Recall that two slopes are perpendicular if one is the negative reciprocal of the other. The reciprocal of $1$ is $\frac{1}{1}$, that is, therefore $1$, $-1$ is the negative reciprocal of $1$. The curves E and S are orthogonal at P.
\qed


\end{document}
