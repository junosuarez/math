\documentclass{article}
\usepackage{fancyhdr}
\usepackage{lastpage}
\usepackage{amsmath}
\usepackage{amssymb}
\usepackage{amsthm}
\usepackage{tabstackengine}
\usepackage[margin=1in]{geometry}
\usepackage{graphicx}
\usepackage{textcomp}
\graphicspath{ {./} }
\TABstackMath
\TABbinary

\pagestyle{fancy}
\setlength\headheight{28pt}
\addtolength{\headheight}{\baselineskip}
\fancyhf{}
\renewcommand{\headrulewidth}{0pt}
\lhead{Math 251 - Shuichi Masuda\\Juno Suárez\\\today}
\rfoot{Page \thepage \hspace{1pt} of \pageref{LastPage}}

\renewcommand\thesubsection{\arabic{subsection}.}
\renewcommand\thesubsubsection{\alph{subsubsection})}

\begin{document}

\section*{Computer Lab \# 5}

\setcounter{subsection}{7}
\subsection{}
Graphically, the points as I plotted them in \# 7 appear to be distributed linearly, rather than sinusoidally. The slopes change periodically (which we would expect for a trigonometric function), but the difference in the slopes as they approach the extrema appears to be linear, as plotted.
If the points were distributed sinusoidally, it would mean that the absolute difference in the slopes decreases as it approaches the extrema.

However, I have reason to be suspicious of this finding, since we know the derivative of
 $\sin \theta$ is $\cos \theta$, and $\cos \theta$ is not a linear function.
  Because the method for choosing $x$ values relied on graphically
  estimating certain geometrically convenient slopes (0\textdegree, 30\textdegree, 45\textdegree, etc.), the
  result is that the $x$ values chosen are not sufficiently close to the extrema to strongly
  suggest either a linear or gradual behavior, when accounting for manual plotting error.

\subsection{}
The s-coordinates represent the slope of the tangent to the function $f(x)=\sin(x)$ at $x$, that is to say, the values of the derivative $f'(x)$ at $x$.

\subsection{}
The graph appears to be the function $\cos \theta$, supported by the fact that it appears to have period $2\pi$, a range of $[-1,1]$, and value of 1 at $x=0$.


\end{document}
